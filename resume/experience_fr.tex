\cvsection{Expériences professionnelles}
\begin{cventries}
  \cventry
    {Développeur backend}
    {Sikana Education (ONG)}
    {Paris, France}
    {Janv. 2016 - Aujourd'hui}
    {
      \begin{cvitems}
        %\item {Sikana is an NGO aiming at producing educative videos to share practical knowledge and skills.}
        \item {Développement de www.sikana.tv, une plateforme diffusant gratuitement des vidéos pédagogiques, avec le framework Symfony.}
        \item {Développement de factory.sikana.tv, une plateforme collaborative où les bénévoles traduisent et sous-titrent les vidéos en 12 langues.}
        \item {Intégration de l'API externe de Sikana avec Learning Equality pour rendre les vidéos disponibles dans les camps de réfugiés.}
        \item {En charge de l'administration des machines virtuelles, de l'assurance qualité, de l'intégration continue et du déploiement.}
      \end{cvitems}
    }
  \cventry
    {Concepteur / Développeur}
    {Gfi Informatique}
    {Orléans, France}
    {Sept. 2015 - Déc. 2015}
    {
      \begin{cvitems}
        \item {Développement d'évolutions pour une application PHP de gestion de base de données. Client : Thalès Air Systems.}
        \item {Conception du système national de gestion de l'assurance maladie pour le Mali. Client : GLI Biométrie.}
      \end{cvitems}
    }
  \cventry
    {Développeur web (en apprentissage, puis en CDI)}
    {Easyflyer (groupe Cimpress)}
    {Orléans, France}
    {Nov. 2013 - Août 2015}
    {
      \begin{cvitems}
        \item {Développement des sites commerçants easyflyer.fr et invite1chef.com sur la plateforme Magento.}
      \end{cvitems}
    }
  \cventry
    {Développeur (stage)}
    {BSAI}
    {Blois, France}
    {Juin 2013 - Août 2013}
    {
      \begin{cvitems}
        \item {Développement d'une application de gestion de base de données avec le framework Code Igniter.}
      \end{cvitems}
    }
\end{cventries}
