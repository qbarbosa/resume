\cvsection{Expériences professionnelles}
\begin{cventries}
  \cventry
    {Tech Lead}
    {LearnEnjoy (Entreprise solidaire d'utilité sociale)}
    {Paris, France}
    {Janv. 2023 - Aujourd'hui}
    {
      \begin{cvitems}
        \item {Référent technique au sein de l'entreprise, membre du comité de direction.}
        \item {Projet "Bien à l'École" : conception et réalisation d'applications pédagogiques avec nos partenaires de l'EdTech.}
        \item {Intéropérabilité : mise à disposition des applications pédagogiques via la norme LTI.}
        \item {Co-construction de features et de la roadmap avec la nouvelle équipe produit. SCRUM mastering.}
        \item {Recrutement et mentoring au sein d'une équipe full stack de 6 personnes.}
        \item {Stack technique : Ruby on Rails, Flutter, ExpressJS, Bash, MySQL, Elastic, Redis, Linux, Vagrant, Ansible.}
      \end{cvitems}
      \vspace{-2.0mm}
    }
  \cventry
    {Développeur backend / Devops}
    {}
    {}
    {Déc. 2021 - Janv. 2023}
    {
      \begin{cvitems}
        \item {Évolutions et maintenance du backoffice destiné aux enseignants, professionnels de santé, et chefs d'établissement.}
        \item {Intéropérabilité : mise à disposition des applications pédagogiques sur le catalogue de l'Éducation Nationale via le portail GAR.}
        \item {Maintenance des serveurs et machines virtuelles de développement.}
        \item {Stack technique : Ruby on Rails, ExpressJS, Bash, MySQL, Elastic, Redis, Linux, Vagrant.}
      \end{cvitems}
    }
  \cventry
    {Lead developer}
    {Believe Music}
    {Paris, France}
    {Juin 2020 - Déc. 2021}
    {
      \begin{cvitems}
        \item {Référent technique de "Data Music", web-app utilisée par les artistes et label managers pour la consultation de données d'écoute.}
        \item {Étude, rétro-documentation, maintenance et évolution de l'application existante.}
        \item {Transformation pas à pas d'un système monolithique en une architecture micro-services (API, BFF, micro-apps).}
        \item {Co-construction de features et d'une roadmap associée avec le produit.}
        \item {Recrutement et mentoring au sein d'une équipe full stack de 9 personnes.}
        \item {Stack technique : PHP, MariaDB, SvelteJS, Java, Docker.}
      \end{cvitems}
      \vspace{-2.0mm}
    }
  \cventry
    {Développeur backend}
    {}
    {}
    {Oct. 2018 - Juin 2020}
    {
      \begin{cvitems}
        \item {Évolutions et maintenance d'un pipeline d'ingestion Big data (+ de 500M de lignes par jour).}
        \item {Transformation du système existant inadapté en un ETL cloud (AWS / Azure).}
        \item {Construction d'APIs permettant l'interaction entre ancien et nouveau système.}
        \item {Stack technique : PHP, Bash, MariaDB, AWS (Lambda, S3, SQS), Azure (Synapse), Airflow.}
      \end{cvitems}
    }
  \cventry
    {Développeur backend / Devops}
    {Sikana Education (ONG)}
    {Paris, France}
    {Janv. 2016 - Mai 2018}
    {
      \begin{cvitems}
        %\item {Sikana is an NGO aiming at producing educative videos to share practical knowledge and skills.}
        \item {Développement de \href{https://sikana.tv}{sikana.tv}, une plateforme diffusant gratuitement des vidéos pédagogiques disponibles en 10 langues.}
        \item {Développement de \href{https://factory.sikana.tv}{factory.sikana.tv}, une plateforme collaborative où les bénévoles traduisent et sous-titrent les vidéos.}
        \item {Administration des machines virtuelles, intégration continue et déploiement. Automatisation de tâches pour la production vidéo.}
        \item {Partenariats techniques avec d'autres ONG internationales (Learning Equality, BSF) pour rendre le contenu accessible hors-ligne.}
        \item {Stack technique : PHP (Symfony 2), Python, MariaDB, Vagrant, Puppet, Git, Capistrano, Jenkins.}
      \end{cvitems}
    }
  \cventry
    {Concepteur / Développeur}
    {Gfi Informatique}
    {Orléans, France}
    {Sept. 2015 - Déc. 2015}
    {
      \begin{cvitems}
        \item {Conception et développement d'applications PHP pour les clients Thalès Air Systems et GLI Biométrie.}
      \end{cvitems}
    }
  \cventry
    {Développeur web (en apprentissage, puis en CDI)}
    {Easyflyer (groupe Cimpress)}
    {Orléans, France}
    {Nov. 2013 - Août 2015}
    {
      \begin{cvitems}
        \item {Développement des sites commerçants \href{https://www.easyflyer.fr}{easyflyer.fr}, \href{https://www.invite1chef.com}{invite1chef.com} et kwikydeco.com sur le framework Magento.}
      \end{cvitems}
    }
%%%
%  \cventry
%    {Développeur (stage)}
%    {BSAI}
%    {Blois, France}
%    {Juin 2013 - Août 2013}
%    {
%      \begin{cvitems}
%        \item {Développement d'une application de gestion de base de données avec le framework Code Igniter.}
%      \end{cvitems}
%    }
%%%
\end{cventries}
